%#!platex --src-specials main.tex

\chapter{はじめに}
\pagenumbering{arabic}     % 絶対必要です.最初の章にのみ記述します.
%プレ 羅列のみ
\section{研究背景}
近年,スマートフォンやタブレットなどの普及によって世界中でタッチパネルが使用されるようになった.
特にスマートフォンは日本でも爆発的に普及しており,2016年時点で56.8\%,つまり
2人に1人がスマートフォンを所持している\cite{soumu}.
さらに,スマートフォン以外の情報端末機器においても
タッチパネルは不可欠なインタフェースとなっている.
しかし,現在のタッチパネルの多くは触覚によるフィードバックが存在しない.研究レベルに
おいては,液晶パネルとの組み合わせを前提とした研究として,
Chubbら\cite{chubb2010shiverpad}によるスクイーズ膜を利用した
摩擦変化による触覚デバイスやKonyoら\cite{konyo2008alternative}
による振動周波数制御と仮想ポインタを利用した触覚デバイスなどがある.
さらに,Wangら\cite{wang2004haptic}は剪断力を利用した
スライディングシステムを開発している.これらの振動刺激は高い
再現性を実現しているが,その振動方向は1次元に限定されている.
これは,既存研究により皮膚構造中の振動刺激を伝える細胞が振動方向の
判別をあまり得意としていないことが判明している
\cite{brisben1999detection}ためである.
そのため,これまでの触覚研究では振動方向はあまり重要視されておらず,
そのほとんどが振動の再生時には1次元の振動に限定されてきた.
しかし,細胞1つ1つに注目してみれば確かに振動方向の判別を得意としてはいないが,
皮膚構造中には振動刺激を伝える細胞のほかにもいくつかの刺激を伝える細胞がある.
また,それらが皮膚中に多数分布しているため,細胞単体のようなミクロな見方ではなく指の腹といった
マクロな見方をすると1次元の振動提示で十分であるのかについては議論の余地があると考えられる.

\section{研究目的とアプローチ}
本研究では,2次元の振動情報を可能な限り正確に再現した際の触覚再現性を検討し,剪断力を用いた
触覚提示における触覚再現性を向上させることを目的とする.これまでに我々は剪断力を用いて
X,Y方向に振動制御可能なテクスチャ提示装置を開発しており,
そのデバイスを用いたテクスチャ感の再現性向上を目的として振動提示手法の検討を行ってきた.
この目的を実現させることで,我々の開発したデバイスに限らず,振動の提示により
触覚を再現するデバイスに対しての応用も可能となる.

本稿では,テクスチャ上を触察した時に発生する振動をはじめとする触覚情報を,
三軸加速度センサを用いて記録し,この振動情報の内,
X軸とY軸の振動情報を用い,これを我々のデバイスで再現する手法の説明する.
さらに,これまでの我々の研究で判明した,再現性の向上が困難なテクスチャに対して再現性を向上
させるために,画像特徴量を利用した振動提示手法を提案し,この装置を利用して実際にユーザに
テクスチャを提示することによる触覚の再現性について実験を行った結果について報告する.
提案手法では,人がテクスチャ上を触察した時に発生する200\ Hz\ 程度の振動をはじめとする触覚情報を,
三軸加速度センサを用いて記録し,記録された情報のX軸とY軸の振動情報を
ディスプレイ上に正確に再現することで,いままで重要視されていなかった振動方向がもたらすリアリティを検証する.
加えて,単純な振動提示では触覚の再現に限界があるテクスチャに対して,画像特徴量
のパラメータの中でも特徴領域の大きさを表すsize情報を抜き出して振動情報に重畳することで触覚再現性の向上を目指す.

\section{本論分の貢献}
 本論文の貢献を以下に示す.\par
 \begin{itemize}
   \item 記録したテクスチャの振動情報の内,
   X軸とY軸方向の振動を正確に提示する手法の提案.
   \item 単純な振動提示では触覚再現性の向上が困難なテクスチャに対して,
   画像特徴量を用いた有効な振動提示手法の提案
   \item 提案手法を用いることでテクスチャの再現性が向上する素材,
   向上しない素材の条件を示した検証結果.
 
 
 \end{itemize}

\section{本論分の構成}
 本稿の構成を以下に示す.まず第2章では本実験に必要な予備知識と関連研究ついて
述べる.第3章では2次元の振動提示手法について詳説し,画像特徴量を用いて振動情報を加工する提案手法についても述べる.
第4章では実際の実験内容と実験結果を示し,その実験結果の考察を述べる.
最後に第5章で本論文のまとめを述べる.
%\section{セクション1}
%Takashitaら\cite{takashita}
%\subsection{サブセクション1}
%Google\cite{google}
% Local Variables:
% TeX-master: "main"
% mode: yatex
% End:
