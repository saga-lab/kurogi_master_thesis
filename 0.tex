%#!platex --src-specials main.tex
%test
\Title{触覚テクスチャ再現性向上のための二次元振動提示手法}
\Author{黒木 詢也}
\Date{31}{2}{7}

%学部の人は次のコメント行の%を外してください.
%\senior  %この\seniorは\Synopsisの前であればどこに書いてもかまいません.

\Synopsis

\begin{Abstract}
%アブスト

近年,スマートフォンなどの普及によって世界中でタッチパネルが使用されている.
しかし,現在のタッチパネルの多くは触覚によるフィードバックが存在しない.
研究レベルにおいては様々な触覚デバイスやシステムが開発されており,
再現性の高い振動刺激による提示手法が提案されている.
しかしその振動方向は1次元に限定されているものがほとんどである.そこで我々は,
テクスチャを触察した際に記録される振動情報をX,Yの二次元方向に正確に再現したときの
触覚再現性を検証する.\par
これまでに我々は剪断力を用いてX,Y方向に振動制御可能なテクスチャ感提示装置を用いて
触覚再現性を向上させる手法を検討してきた.
本稿では,テクスチャ上を触察した時に発生する振動情報を,
三軸加速度センサを用いて記録し,この振動情報の内,X軸とY軸の振動情報を用い,
これを我々のデバイスで再現する手法と,これまでの我々の研究から触覚の再現が困難であることが分かっている
テクスチャに対して,触覚再現性を向上させるための手法として,画像特徴量を利用した振動提示手法を提案を提案する.また,この装置を利用して実際にユーザにテクスチャを提示することによる触覚の再現性について実験を行った結果について報告する.


\end{Abstract}

%
% これより下は学部(卒業論文を書く人)には関係ありません.
%
\title{The Rendering Method of 2-Dimensional Vibration Presentation
for Improving Fidelity of Haptic Texture}
\author{Junya Kurogi}
\endate{February}{14}{2}

\synopsis

\begin{abstract}
In recent years, touchscreens have been used all over the world, however, most of them are without realistic haptic feedback. Some of them have feedback, but most of them have vibration direction limited to one direction. Here we propose a novel rendering method for direction-controlled 2-dimensional vibration display to present texture information. In this paper, we proposed a dimension-controlled rendering method of texture information which capable of vibration control in the X and Y-axis precisely by using lateral force. Further, to improve the fidelity for large-scaled texture, we proposed to combine image features information of the textures. We held an experiment to evaluate the fidelity of the proposed method. The result shows that the proposed method can present randomized textures and periodic textures more precisely than the conventional method.
\end{abstract}










% 修士論文の論文概要

% 修士論文については和文と英文の論文概要を次の要領で作成し,【5】のb.2として下
% さい。英文で本文を記述した場合も,論文概要は和文,英文の両方で作成することが望ま
% しい(詳細は指導教員の指示に従って下さい)。

% 1.論文概要(和文)の形式

%     a.修士論文と同じ体裁で作った表紙(図2)を付けます。ただし,題目と氏名の間
%     に,「論文概要」と書き添えて下さい。

%     b.研究の目的,論文全体のあらまし,各章の内容(簡単に),結論(やや詳し
%     く),得られた成果の意義を,この順序で3~5ページ程度にまとめて下さい。

% 2.論文概要(英文)の形式

%    a.修士論文と同じ体裁で作った表紙(図2)を英文で記載して付けます。ただし,題
%    目と氏名の間に, Synopsis と書き添えて下さい

%    b.研究の目的,論文全体のあらまし,結論を,100~300 語程度にまとめて下さい。


% 3.和文,英文ともに目次は付けないで下さい。また,原則として,図,表,式を用いな
% いで下さい。
