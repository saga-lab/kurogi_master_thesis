%#!platex --src-specials main.tex
\chapter{おわりに}

\section{結論}
本論文では,振動を用いて仮想触覚を提示する際の振動方向に注目した.
既存の触覚研究では振動の方向はあまり重要視されておらず,振動再現の
際には1次元の振動に限定されることが多かった.このことから本研究では,
2方向に指を動かしたときに得られる加速度を記録し,適切にそれらの和をとることで,二次元方向の振動情報を
ディスプレイ上に正確に提示する新たな手法を提案した.また,実際にユーザに
提示したときのテクスチャ再現性について検証した.\par
実験の結果,本論文の提案手法は,比較的硬い素材かつランダムな空間周波数をもつ
ものにおいて,既存手法より再現性の高いテクスチャを提示することが
できることが分かった.また,提案手法が苦手としている一定の空間周波数をもつテクスチャ
に対して再現性高くテクスチャを提示するために画像特徴量を用いる手法を提案し,この有用性に
ついても確認した.
\par
これらの実験結果を踏まえた上で,本研究の貢献を以下に示す.
\begin{itemize}
  \item 記録したテクスチャの振動情報の内,
  X軸とY軸方向の振動を正確に提示する手法の提案.
 \item 画像特徴量の重畳を用いた新たな振動提示手法の提案
  \item 提案手法を用いることでテクスチャの再現性が向上する素材,
  向上しない素材の条件を示した検証結果.
  
\end{itemize}
今回の実験結果より,不規則な空間周波数をもつ硬い素材を
提示する際に,X軸とY軸方向の振動を正確に再現することが有効である
ことを示すことができた.
また単純な振動提示では提示が困難なテクスチャに対して
再現性を向上させる手法を提案し,その有用性を示すことができた.
これらの手法を提案できたことで今回用いたデバイスに限らず,剪断力を用いてディスプレイ上により再現度の高いテクスチャを
提示する研究全般に貢献できたと考える.


% Local Variables:
% TeX-master: "main"
% mode: yatex
% End:
